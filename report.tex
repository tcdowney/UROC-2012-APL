\input ConTeXtLPMacros

\definefontsynonym[APL][Apl385]
\setupbodyfont[xits]
\definefont[tt][APL sa 1]

\setuppapersize[letter][letter]
\setupwhitespace[medium]

\starttext
\startfrontmatter
\title{HPAPL: The Compiler}

\vfill
Copyright $\copyright$ 2012 \hfill\break
Aaron W. Hsu $⟨${\tt arcfide@sacrideo.us}$⟩$,\hfill\break
Brett Schrank $⟨${\tt bschrank@indiana.edu}$⟩$,\hfill\break
Timothy Downey $⟨${\tt tcdowney@indiana.edu}$⟩$.

Permission to use, copy, modify, and distribute this software for any
purpose with or without fee is hereby granted, provided that the above
copyright notice and this permission notice appear in all copies.

THE SOFTWARE IS PROVIDED "AS IS" AND THE AUTHOR DISCLAIMS ALL WARRANTIES
WITH REGARD TO THIS SOFTWARE INCLUDING ALL IMPLIED WARRANTIES OF
MERCHANTABILITY AND FITNESS. IN NO EVENT SHALL THE AUTHOR BE LIABLE FOR
ANY SPECIAL, DIRECT, INDIRECT, OR CONSEQUENTIAL DAMAGES OR ANY DAMAGES
WHATSOEVER RESULTING FROM LOSS OF USE, DATA OR PROFITS, WHETHER IN AN
ACTION OF CONTRACT, NEGLIGENCE OR OTHER TORTIOUS ACTION, ARISING OUT OF
OR IN CONNECTION WITH THE USE OR PERFORMANCE OF THIS SOFTWARE.

\completecontent
\stopfrontmatter

\startbodymatter

\chapter{Introduction}

This is an implementation of the NAS Parallel Benchmarks for Dyalog APL.
At the moment, it includes implementations of the IS, EP, and CG 
micro-benchmarks.  Our intention in creating these benchmarks is to 
establish a baseline of performance for current industrial strength 
implementations of APL, against which new research in APL optimization 
may be judged.  The NPB benchmark suite is a straigthforward, but 
well accepted set of benchmarks for testing distributed and parallel 
computation, which we anticipate will be the focus of modern APL 
optimization efforts.

Each benchmark is defined in its own namespace named according to its 
two-letter identifier. The following namespace can be used to load in 
individual benchmarks into an APL session.

\defchunk{Benchmark Loader}
:Namespace Loader
  benchfile←'./dyalog_npb.tex'
  ∇ Load BNM;CN;FN
    #.⎕SE.SALT.Load './ConTeXtLP -Target=#'
    CN←BNM,' Namespace'
    FN←'./',BNM,'.dyalog'
    CN #.ConTeXtLP.Tangle benchfile FN
    #.⎕SE.SALT.Load FN,' -Target=#'
  ∇
:EndNamespace
\stopchunk

To initially bootstrap the system, you can use the provided 
“boostrap.dyalog” file.  Otherwise, you can manually tangle the loader 
namespace using the following set of session commands.

\starttyping
]LOAD ./ConTeXtLP
'Benchmark Loader' ConTeXtLP.Tangle './dyalog_npb.tex' './loader.dyalog'
]LOAD ./loader
\stoptyping

\chapter{Random Number Generation}

Most of the NPB benchmarks require a specific style and implemenation 
of a random number generator in order to get deterministic results. 
Specifically, it is a linear congruential generator that obeys the following 
sequence for producing the next seeds.

\startformula
x_1 = ax_0\ (\mod 2^{46})
\stopformula

And the random number is generated on the range $(0,1)$ by computing 
$2^{-46}x$. We implement this as a class to make it easier to encapsulate 
the state data. You create a random generator by passing either a seed 
or a seed and an $a$ multiplier, which by default is $5^{13}$. 

\defchunk{Random Namespace}
:Class Random
   ⎕IO ⎕ML←0 0
   Seed A V←0 0,2*24
   
   ∇Make S
    :Access Public
    :Implements Constructor
    ⎕SIGNAL (1≡⊃⍴S)/11
    Seed A←S,5*13
   ∇
   
   ∇MakeWithA(s a)
    :Access Public
    :Implements Constructor
    Seed A←s a
   ∇
   
   Mul←{(2*46)|V⊥(V|A+.×B),[¯0.5](0⌷A←⌽V V⊤⍺)×1⌷B←V V⊤⍵}
	
   ∇R←RandI N
    :Access Public
    R←Seed
	0 0⍴({⍵ Mul ⍵⊣R,←⍵ Mul R}⍣(⌈2⍟N))A
	Seed←⊢/R←N↑R⊣R,←A Mul⊢/R←1↓R
	R←R×2*¯46
   ∇
:EndClass
\stopchunk

\chapter{Timing Benchmark Results}

To unify the timing mechanism for all of these benchmarks, we provide 
a class for timing.

\defchunk{Timer Namespace}
:Class Timer
    ⎕IO ⎕ML←0 0
    
    start←0
    end←0
    
    ∇ T←CurrentTime
      :Access Public
      T←(÷1000)×24 60 60 1000⊥3↓⎕TS
    ∇
    
    ∇ Start
      :Access Public
      start←CurrentTime
    ∇
    
    ∇ End
      :Access Public
      end←CurrentTime
    ∇
    
    ∇ Extend
      :Access Public
      End
    ∇
    
    ∇ R←Spent
      :Access Public
      R←end-start
    ∇
:EndClass
\stopchunk

\chapter{IS: Integer Sort}

The purpose of the IS benchmark is to test parallel sorting operations 
and uniform distribution of keys in memory.  To accomplish this we generate
N keys using the random number generator defined in Section 2.  Timing begins,
and sorting is accomplished by computing the rank of each key.  The rank of a 
key is defined as the index that the key would have in the sorted list.  During
ranking partial verification tests are performed to ensure the integrity of the 
procedure.  At the conclusion of timing, a full verification of the sort is
performed.

\defchunk
ClassA←(2*23)(2*19)314159265 10
N Bmax startSeed Imax←class
:If class≡ClassA
	partialKeyIndex←211237 662041 5336171 3642833 4250760  
	partialKeyRank←104 17523 123928 8288932 8388264        
:EndIf
\stopchunk

To run the IS benchmark, a class definition is passed as its right argument.
\textit{ClassA} is defined in accordance with the NAS Parallel Benchmark specifications,
where \textit{N} is the number of random keys generated, \textit{Bmax} is an index used when
modifying the sequence of keys, \textit{startSeed} is the beginning seed for the random
number generator, and \textit{Imax} is the number of iterations for modifying the
sequence of keys.

This version of the benchmark currently works only with \textit{ClassA}, but it could be
modified to include other class specifications.  The following :If statement
provides the necessary values for the partial verification of \textit{ClassA}.  See
Table 2.8 and Figure 2.9 in the NPB documentation for further information on
what these values mean.
 
\defchunk
Gen←⎕NEW #.Random(startSeed,5*13)
keys←Gen.RandI N
\stopchunk

Here generate \textit{keys}, an array of random numbers of \textit{N} length.

\defchunk
Ts←24 60 60 1000⊥3↓⎕TS
:For i :In ⍳Imax
	keys[i Imax+i]←i Bmax-i                                
	R←⍋keys                                               
	:For j :In ⍳5                                         
		:If j<3
			:If R[partialKeyIndex[j]]≠partialKeyRank[j]+i
				⎕←'Iteration',i,': Partial verification failed.'
			:EndIf
		:Else
			:If R[partialKeyIndex[j]]≠partialKeyRank[j]-i
				⎕←'Iteration',i,': Partial verification failed.'
			:EndIf
		:EndIf
	:EndFor
:EndFor
\stopchunk

Timing begins at this point.  First a time stamp is acquired with the ⎕TS
system function.  However, it is provided in a format that includes 
superfluous information such as the month and year.  We drop the first three
entries and combine the hours, minutes, seconds, and milliseconds into a single
value using the Decode operator.  This saves the start time, \textit{Ts}.

Next we modify the sequence of keys through the following method:

\startformula
keys_{i} \leftarrow i
keys_{i + Imax} \leftarrow (Bmax - i)
\stopformula

This is accomplished through the use of a multiple index assignment.

Then we conveniently compute the ranks of the keys with the GradeUp 
primitive and assign them to the vector \textit{R}.

Within each iteration, we also perform a partial verification test.  This test
checks whether the rank at a particular index in \textit{R} is equal to the 
provided reference rank.  The values for this index are given in 
\textit{partialKeyIndex}, while the comparison values are provided in the 
corresponding \textit{partialKeyRank} array.  There are five total values to
test; we accomplish this through a simple :For loop.  Refer once more to 
Table 2.8 in the NPB documentation for the values used in partial verification.

\defchunk
Te←24 60 60 1000⊥3↓⎕TS                                    
⎕←'Time taken: ',(⍕(Te-Ts)÷1000),' seconds.'
:If ∧/2≤/keys[R]                                           
	⎕←'Full verification suceeded.'
:Else
	⎕←'Full verification failed.'
:EndIf
\stopchunk

We record the end time, \textit{Te}, in the same manner as \textit{Ts}.  We
then subtract \textit{Te} from \textit{Ts} and divide it by 1000 to return
the total number of seconds elapsed.

Once timing is complete, a full verification test is performed.  The keys are
rearranged in their ranking order, which is easily done by indexing the ranks
contained in \textit{R}.  Next we compare each pair of keys, testing to see
whether

\startformula
keys_{i} \leq keys_{i+1} for every keys_{i} from i=0...N-2
\stopformula

This is accomplished by performing a pairwise reduction of the less-than-or-equal-
to primitive over the ranked keys. This creates a new array of boolean values
which express whether each pair satisfied the inequality.  A simple "and" reduction
determines whether every pair satisfies the inequality, or that the keys are in fact
in sorted order.

Our first attempt at full verification did not make use of the pairwise reduction,
relying instead upon the simple reduction.  This, however, does not work due to 
the right-to-left order of operations in APL.

The entire IS benchmark is provided below for full viewing:

\defchunk{IS Namespace}
:Namespace IS

    ClassA←(2*23)(2*19)314159265 10

    ∇ Run class;N;Bmax;startSeed;Imax;keys;Seed;Ts;i;R;Te;partialKeyIndex;partialKeyRank;j;Gen
      ⍝ Parameter values used by various classes,
      ⍝ See Figure 2.9 of NPB IS benchmark document.
      N Bmax startSeed Imax←class
      :If class≡ClassA
          partialKeyIndex←211237 662041 5336171 3642833 4250760  ⍝ Values to be used for partial
          partialKeyRank←104 17523 123928 8288932 8388264        ⍝ verification as specified in Table 2.8
      :EndIf
      Gen←⎕NEW #.Random(startSeed,5*13)
      keys←Gen.RandI N                                           ⍝ Generate N keys
      Ts←24 60 60 1000⊥3↓⎕TS                                     ⍝ Begin timing
      :For i :In ⍳Imax
          keys[i Imax+i]←i Bmax-i                                ⍝ Modify sequence of keys
          R←⍋keys                                                ⍝ Compute the ranks of keys
          :For j :In ⍳5                                          ⍝ Partial verification test
              :If j<3
                  :If R[partialKeyIndex[j]]≠partialKeyRank[j]+i
                      ⎕←'Iteration',i,': Partial verification failed.'
                  :EndIf
              :Else
                  :If R[partialKeyIndex[j]]≠partialKeyRank[j]-i
                      ⎕←'Iteration',i,': Partial verification failed.'
                  :EndIf
              :EndIf
          :EndFor
      :EndFor
      Te←24 60 60 1000⊥3↓⎕TS                                     ⍝ End timing
      ⎕←'Time taken: ',(⍕(Te-Ts)÷1000),' seconds.'
      :If ∧/2≤/keys[R]                                           ⍝ Full verification test
          ⎕←'Full verification suceeded.'
      :Else
          ⎕←'Full verification failed.'
      :EndIf
    ∇

:EndNamespace 
\stopchunk

\chapter{EP: Embarassingly Parallel}

The purpose of the EP benchmark is to generate pairs of Gaussian random
deviates and tabulate the number of pairs in successive square annuli.
According to the NPB documentation, this benchmark is related to Monte 
Carlo simulations.  Due to the memory constraints caused by the magnitude 
of random numbers generated and the various arrays required to store 
information, it was necessary to break the task of the benchmark into 
multiple blocks.  In fact, the name of the benchmark, "Embarassingly Parallel," 
refers to the ability of this problem to be independently computed on multiple
processors.

This problem was beneficial to write in APL, for the algorithm 
provided in the NPB documentation translated rather seamlessly to code.

First we generate an array of random numbers (once again using the random 
number generator provided in Section 2) of length 2n.  This array is used 
to form two separate arrays, \textit{x} and \textit{y}, which represent pairs
such that, for j in range 1 to n,

\startformula
x_{j} = 2r_{2j-1}-1
y_{j} = 2r_{2j}-1
\stopformula

This ensures that these pairs are uniformly distributed on the interval (-1,1).

Now we create a new array \textit{tf} that contains boolean values representing 
whether all pairs in \textit{x} and \textit{y} satisfy the following inequality:

\startformula 
t = x^{2} + y^{2} \leq 1
\stopformula

We use \textit{tf} to eliminate all invalid pairs from \textit{x}, \textit{y},
and \textit{t}.

Next we form the Gaussian random deviates from these remaining valid pairs
using the following formula:

\startformula
X = x\sqrt[2]{(-2ln t)/t}
Y = y\sqrt[2]{(-2ln t)/t}
\stopformula

The above instructions were to be completed in each block.  The 
function \textit{Run} takes as its left argument the block size and 
computes the benchmark by splitting it up into separate sections.  
Again, this was necessary due to the memory-intensive nature of 
each array used in the program.

The entire EP benchmark is provided below for full viewing.

\defchunk{EP Namespace}
:Namespace EP
    ⎕IO ⎕ML←1 0
    
    Gen←⍬
    
    DoBlock←{
        n←⍵
        r←Gen.RandI 2×n
        x y←⊂[2]¯1+2×r[¯1 0+[1]2 n⍴2×⍳n]
        tf←1≥t←⊃+/x y*2
        x y t←(⊂tf)/¨x y t
        X Y←x y×⊂((¯2×⍟t)÷t)*÷2
        Q←+/(l∘.≤M)∧(l+1)∘.>M←(|X)⌈|Y⊣l←¯1+⍳10
        (+/¨X Y),Q
    }
    
    ∇R←BS Run N;Time;C
       Gen←⎕NEW #.Random(271828183,5*13)
       Time←⎕NEW #.Timer
       C←⌊N÷BS
       Time.Start
       R←BS({⍵+DoBlock ⍺}⍣C)12⍴0
       R+←{0≠⍵: DoBlock ⍵ ⋄ 12⍴0} N-BS×C
       Time.End
       ⎕←'Running took ',(⍕Time.Spent),' seconds.'
    ∇

:EndNamespace
\stopchunk

\chapter{CG: Conjugate Gradient}

\defchunk{CG Namespace}
:Namespace CG
⍝ === VARIABLES ===

    ClassA←14000 15 11 20

    ClassSample←1400 15 7 10


⍝ === End of variables definition ===

    ⎕IO ⎕ML ⎕WX ⎕RL←0 0 3 2090219546

    ∇ ZRN←X CgSparse A;R;Rho;P;I;Q;Alpha;RhoZ;Beta;Z
      Z←0 ⋄ R←X ⋄ Rho←R+.×R ⋄ P←R
      :For I :In ⍳25
          Q←A SMVP P
          Alpha←Rho÷P+.×Q
          Z←Z+Alpha×P
          RhoZ←Rho
          R←R-Alpha×Q
          Rho←R+.×R
          Beta←Rho÷RhoZ
          P←R+Beta×P
      :EndFor
      R←X-A SMVP Z
      ZRN←Z((R+.×R)*0.5)
    ∇

    ∇ A←MakeSA(N NZ L);W;R;Xi;Xv;I;IV;RN;CI;DV;Ts;Te;S;Av;Ai;At;F
     ⍝ Generate a test matrix of order N as the weighted sum of N outer products of
     ⍝ random sparse vectors.
      ⎕IO←0 ⋄ Ts←⎕TS
      R←0.1*÷N-1                    ⍝ Geometric ratio for W
      Ai Av←⍬ ⍬                     ⍝ Initialize empty sparse matrix A
      :For I :In ⍳N
          Xi←(I≠Xi)/Xi←NZ?N         ⍝ NZ nonzeros maybe with no X_i
          Xv←(2*¯30)×?(⍴Xi)⍴2*30    ⍝ Random values (0..1)
          Xi,←I ⋄ Xv,←0.5           ⍝ X_i is 0.5
          Av,←(R*I)×,Xv∘.×Xv        ⍝ Catenate weighted outer product
          At←((⍴Xi)*2)⍴Xi           ⍝ Nonzero row/column pattern
          Ai,←At+N×,⍉(2⍴⍴Xi)⍴At     ⍝ Catenate respective ravel indices
      :EndFor
      S←⍋Ai ⋄ Ai←Ai[S] ⋄ Av←Av[S]
      S←(1⌽Ai)≠Ai
      IV←S/Ai
      DV←S/+\Av
      DV←(⊃DV),1↓DV-¯1⌽DV
      CI←N|IV                              ⍝ Extract the Column indices
      RN←N⍴0 ⋄ RN[(IV-CI)÷N]+←1            ⍝ Build the NZ row vector
      DV[((RN/⍳⍴RN)=CI)/⍳⍴DV]+←0.1         ⍝ Diagonal gets additional 0.1
      CI←N|CI-L                            ⍝ Shift the diagonal
      A←RN CI DV                           ⍝ A with goodies
      Te←⎕TS
      ⎕←'Made array in ',(⍕Ts #.UTIL.TimerSecs Te),' seconds.'
    ∇

    ∇ R←M SMVP V;S;NZ;CI;DV;B;⎕IO
      ⎕IO←0
     ⍝ Sparse Matrix, Dense vector product, inspired by
     ⍝  Hendriks, Ferdinand and Wai-Mee Ching.
     ⍝  "Sparse Matrix Technology Tools in APL."
     ⍝  1990 ACM.
     ⍝ NZ CI DV←M                ⍝ Explode the matrix
     ⍝ R←DV×V[CI]
     ⍝ R←0,+\R            ⍝ Do the product
     ⍝ R←R[(0≠NZ)×+\NZ]          ⍝ Get desired sums
     ⍝ S←(0≠R)/⍳⍴R               ⍝ Select the nonzero results
     ⍝ R[S]←R[⊃S],1↓R[S]-¯1⌽R[S] ⍝ Subtract excess
      ⎕IO←0
      NZ CI DV←M
      B←(+/NZ)⍴0
      B[¯1++\NZ]←1
      R←(×NZ)\-2-/0,B/+\DV×V[CI]
    ∇

    ∇ (Cg Solve)(N NI NZ L);A;X;I;Z;R;Rn;Zeta;Ts;Te;⎕IO
      ⎕IO←0
      A←MakeSA N NZ L
      X←N⍴1
      ⎕←'Iteration        ∥R∥                 Zeta'
      ⎕←'─────────────────────────────────────────────────────'
      Ts←⎕TS
      :For I :In 1+⍳NI
          Z Rn←X Cg A
          Zeta←L+÷X+.×Z
          ⎕←9 0 20 ¯7 24 13⍕I Rn Zeta
          X←Z÷(Z+.×Z)*0.5
      :EndFor
      Te←⎕TS
      ⎕←''
      ⎕←'Time taken: ',(⍕Ts #.UTIL.TimerSecs Te),' seconds.'
    ∇

:EndNamespace 
\stopchunk


\stopbodymatter

\startappendices

\stopappendices

\startbackmatter
\completeindex
\stopbackmatter

\stoptext
